%% \documentclass{article}
%% \usepackage{tikz}
%% \usetikzlibrary{matrix, fit}
%% \usetikzlibrary{backgrounds}

%% \usepackage{nicematrix}
%% %\NiceMatrixOptions{
%% %code-for-first-row = \color{blue} ,
%% %code-for-last-row = \color{blue} ,
%% %code-for-first-col = \color{blue} ,
%% %code-for-last-col = \color{blue}
%% %}

%% \newcommand\x{\times}





%% \begin{document}
%% \begin{tikzpicture}
%% $\begin{pNiceMatrix}[first-row,last-row,first-col,last-col]
%%     & C_1 & C_2 & C_3 & C_4       \\
%% L_1 & a   & b   & c   & d   & L_1 \\
%% L_2 & e   & f   & g   & h   & L_2 \\
%% L_3 & i   & j   & k   & l   & L_3 \\
%%     & C_1 & C_2 & C_3 & C_4       \\
%% \end{pNiceMatrix}$


%%     \matrix[
%%         matrix of math nodes,
%%         row sep=.5ex,
%%         column sep=.5ex,
%%         left delimiter=(,right delimiter=),
%%         nodes={text width=.75em, text height=1.75ex, text depth=.5ex, align=center}
%%         ] (m) 
%%         {
%%         \x & \x & \x & \x \\
%%         0  & \x & \x & \x \\
%%         0  & 0  & \x & \x \\
%%         0  & 0  & 0  & \x \\
%%         1  & 1  & 1  & 1 \\
%%         };
%%         \begin{scope}[on background layer]
%%             \node[fit=(m-2-1)(m-2-4), draw=green!30, fill=green!30, rounded corners] {};
%%             \node[fit=(m-1-3)(m-5-3), draw=green!30, fill=green!30, rounded corners] {};
%%             \node[fit=(m-2-3), fill=green] {};
%%         \end{scope} 
%% \end{tikzpicture}
%% \end{document}



%% \documentclass{article}
%% \usepackage{kbordermatrix} % include package @ document preamble
%% \renewcommand{\kbldelim}{(} % change default array delimiters to parentheses
%% \renewcommand{\kbrdelim}{)}

%% \begin{document}
%% % ...

%% \[
%% \kbordermatrix{
%%     \mbox{corner_text}&\alpha & \beta & \gamma & \delta \\ % column indices
%%     1 & t   & 1-t &  -1 &  0  &  0  \\
%%     2 & 0   &  t  & 1-t &  -1 &  0  \\
%%     3 & 0   &  0  &  t  & 1-t &  -1 \\
%%     4 & -1  &  0  &  0  &  t  & 1-t \\
%%     5 & 1-t &  -1 &  0  &  0  &  t
%%     % 1, 2, 3, 4, 5 are row indices
%% }
%% \]
%% \end{document}


\documentclass[12pt]{report}
\usepackage{blkarray}
\usepackage{amsmath}

\begin{document}

\[
\begin{blockarray}{cccccc}
a & b & c & d & e \\
\begin{block}{(ccccc)c}
  1 & 1 & 1 & 1 & 1 & f \\
  0 & 1 & 0 & 0 & 1 & g \\
  0 & 0 & 1 & 0 & 1 & h \\
  0 & 0 & 0 & 1 & 1 & i \\
  0 & 0 & 0 & 0 & 1 & j \\
\end{block}
\end{blockarray}
 \]

\end{document}